%% BioMed_Central_Tex_Template_v1.06
%%                                      %
%  bmc_article.tex            ver: 1.06 %
%                                       %

%%IMPORTANT: do not delete the first line of this template
%%It must be present to enable the BMC Submission system to
%%recognise this template!!

%%%%%%%%%%%%%%%%%%%%%%%%%%%%%%%%%%%%%%%%%
%%                                     %%
%%  LaTeX template for BioMed Central  %%
%%     journal article submissions     %%
%%                                     %%
%%          <8 June 2012>              %%
%%                                     %%
%%                                     %%
%%%%%%%%%%%%%%%%%%%%%%%%%%%%%%%%%%%%%%%%%


%%%%%%%%%%%%%%%%%%%%%%%%%%%%%%%%%%%%%%%%%%%%%%%%%%%%%%%%%%%%%%%%%%%%%
%%                                                                 %%
%% For instructions on how to fill out this Tex template           %%
%% document please refer to Readme.html and the instructions for   %%
%% authors page on the biomed central website                      %%
%% http://www.biomedcentral.com/info/authors/                      %%
%%                                                                 %%
%% Please do not use \input{...} to include other tex files.       %%
%% Submit your LaTeX manuscript as one .tex document.              %%
%%                                                                 %%
%% All additional figures and files should be attached             %%
%% separately and not embedded in the \TeX\ document itself.       %%
%%                                                                 %%
%% BioMed Central currently use the MikTex distribution of         %%
%% TeX for Windows) of TeX and LaTeX.  This is available from      %%
%% http://www.miktex.org                                           %%
%%                                                                 %%
%%%%%%%%%%%%%%%%%%%%%%%%%%%%%%%%%%%%%%%%%%%%%%%%%%%%%%%%%%%%%%%%%%%%%

%%% additional documentclass options:
%  [doublespacing]
%  [linenumbers]   - put the line numbers on margins

%%% loading packages, author definitions

%\documentclass[twocolumn]{bmcart}% uncomment this for twocolumn layout and comment line below
\documentclass{bmcart}

%%% Load packages
%\usepackage{amsthm,amsmath}
%\RequirePackage{natbib}
%\RequirePackage{hyperref}
\usepackage[utf8]{inputenc} %unicode support
%\usepackage[applemac]{inputenc} %applemac support if unicode package fails
%\usepackage[latin1]{inputenc} %UNIX support if unicode package fails


\usepackage{graphicx,ifthen}
\usepackage{wrapfig}
\usepackage{color,calc,graphicx,soul,lipsum}
\usepackage{amssymb, latexsym}
\usepackage{mathrsfs}
\usepackage{algorithm}
\usepackage{algorithmic}

%%%%%%%%%%%%%%%%%%%%%%%%%%%%%%%%%%%%%%%%%%%%%%%%%
%%                                             %%
%%  If you wish to display your graphics for   %%
%%  your own use using includegraphic or       %%
%%  includegraphics, then comment out the      %%
%%  following two lines of code.               %%
%%  NB: These line *must* be included when     %%
%%  submitting to BMC.                         %%
%%  All figure files must be submitted as      %%
%%  separate graphics through the BMC          %%
%%  submission process, not included in the    %%
%%  submitted article.                         %%
%%                                             %%
%%%%%%%%%%%%%%%%%%%%%%%%%%%%%%%%%%%%%%%%%%%%%%%%%


%\def\includegraphic{}
%\def\includegraphics{}



%%% Put your definitions there:
\startlocaldefs
%%%%%%%%%% special commands %%%%%%%%%%%%%%%%%%%%%%%%%%
\def\mtry{{\sl mtry }}
\def\ntree{{\sl ntree }}
\def\nsplit{{\sl nsplit }}
\def\nodesize{{\sl nodesize }}
\def\RF{\textrm{RF}}
\def\RFopt{RFopt}
\def\X{{\bf X}}
%%%%%Ralph Smith Formal Script%%%%%%%%
\def\RSFS{\mathscr}
\def\nn{{\RSFS N}}
%%%%%%%%%% color %%%%%%%%%%%%%%%%%%%%%%%%%%
\newcommand{\black}{\textcolor{black}}
\newcommand{\red}{\textcolor{red}}
\newcommand{\green}{\textcolor{green}}
\newcommand{\blue}{\textcolor{blue}}
\newcommand{\gray}{\textcolor{gray}}
\newcommand{\cyan}{\textcolor{cyan}}
\definecolor{aqua}{rgb}{0.0, 1.0, 1.0}
\newcommand{\aqua}{\textcolor{aqua}}
\newcommand{\yellow}{\textcolor{yellow}}
\endlocaldefs


%%% Begin ...
\begin{document}

%%% Declare graphical extensions
\DeclareGraphicsExtensions{.gif,.pdf,.png,.jpg,.tiff,.eps}

%%% Start of article front matter
\begin{frontmatter}

\begin{fmbox}
\dochead{Methodologies}

%%%%%%%%%%%%%%%%%%%%%%%%%%%%%%%%%%%%%%%%%%%%%%
%%                                          %%
%% Enter the title of your article here     %%
%%                                          %%
%%%%%%%%%%%%%%%%%%%%%%%%%%%%%%%%%%%%%%%%%%%%%%

\title{Synthetic Learning Machines}

%%%%%%%%%%%%%%%%%%%%%%%%%%%%%%%%%%%%%%%%%%%%%%
%%                                          %%
%% Enter the authors here                   %%
%%                                          %%
%% Specify information, if available,       %%
%% in the form:                             %%
%%   <key>={<id1>,<id2>}                    %%
%%   <key>=                                 %%
%% Comment or delete the keys which are     %%
%% not used. Repeat \author command as much %%
%% as required.                             %%
%%                                          %%
%%%%%%%%%%%%%%%%%%%%%%%%%%%%%%%%%%%%%%%%%%%%%%

\author[
   addressref={aff1},                   % id's of addresses, e.g. {aff1,aff2}
   corref={aff1},                       % id of corresponding address, if any
   noteref={n1},                        % id's of article notes, if any
   email={hemant.ishwaran@gmail.com}   % email address
]{\inits{HI}\fnm{Hemant} \snm{Ishwaran}}
\author[
   addressref={aff2},
   noteref={n1},
   email={jmalley@mail.nih.gov}
]{\inits{JDM}\fnm{James D} \snm{Malley}}

%%%%%%%%%%%%%%%%%%%%%%%%%%%%%%%%%%%%%%%%%%%%%%
%%                                          %%
%% Enter the authors' addresses here        %%
%%                                          %%
%% Repeat \address commands as much as      %%
%% required.                                %%
%%                                          %%
%%%%%%%%%%%%%%%%%%%%%%%%%%%%%%%%%%%%%%%%%%%%%%

\address[id=aff1]{%                           % unique id
  \orgname{Division of Biostatistics, University of Miami}, % university, etc
  \street{1120 NW 14th Street},                     %
  \postcode{Miami FL, 33136},                                % post or zip code
  \cny{USA}                                   % country
}
\address[id=aff2]{%
  \orgname{Center for Information Technology, National Institutes
    of Health},
  \city{Bethesda MD, 20892}, 
  \cny{USA},
}

%%%%%%%%%%%%%%%%%%%%%%%%%%%%%%%%%%%%%%%%%%%%%%
%%                                          %%
%% Enter short notes here                   %%
%%                                          %%
%% Short notes will be after addresses      %%
%% on first page.                           %%
%%                                          %%
%%%%%%%%%%%%%%%%%%%%%%%%%%%%%%%%%%%%%%%%%%%%%%

\begin{artnotes}
%\note{Sample of title note}     % note to the article
\note[id=n1]{Equal contributor} % note, connected to author
\end{artnotes}

\end{fmbox}% comment this for two column layout

%%%%%%%%%%%%%%%%%%%%%%%%%%%%%%%%%%%%%%%%%%%%%%
%%                                          %%
%% The Abstract begins here                 %%
%%                                          %%
%% Please refer to the Instructions for     %%
%% authors on http://www.biomedcentral.com  %%
%% and include the section headings         %%
%% accordingly for your article type.       %%
%%                                          %%
%%%%%%%%%%%%%%%%%%%%%%%%%%%%%%%%%%%%%%%%%%%%%%

\begin{abstractbox}

\begin{abstract} % abstract
\parttitle{Background} Using a collection of different
terminal nodesize constructed random forests, each generating a synthetic feature, a
synthetic random forest is defined as a kind of hyperforest,
calculated using the new input synthetic features, along with the
original features.

\parttitle{Results} 
Using a large collection of regression and multiclass datasets we
show that synthetic random forests outperforms both conventional
random forests and the optimized forest from the regresssion
portfolio.

\parttitle{Conclusions} 
Synthetic forests removes the need for tuning random forests with no
additional effort on the part of the researcher.  Importantly, the
synthetic forest does this with evidently no loss in prediction
compared to a well-optimized single random forest.


\end{abstract}

%%%%%%%%%%%%%%%%%%%%%%%%%%%%%%%%%%%%%%%%%%%%%%
%%                                          %%
%% The keywords begin here                  %%
%%                                          %%
%% Put each keyword in separate \kwd{}.     %%
%%                                          %%
%%%%%%%%%%%%%%%%%%%%%%%%%%%%%%%%%%%%%%%%%%%%%%

\begin{keyword}
\kwd{Machine}
\kwd{Nodesize}
\kwd{Random Forest}
\kwd{Trees}
\kwd{Synthetic feature}
\end{keyword}

% MSC classifications codes, if any
%\begin{keyword}[class=AMS]
%\kwd[Primary ]{}
%\kwd{}
%\kwd[; secondary ]{}
%\end{keyword}

\end{abstractbox}
%
%\end{fmbox}% uncomment this for twcolumn layout

\end{frontmatter}

%%%%%%%%%%%%%%%%%%%%%%%%%%%%%%%%%%%%%%%%%%%%%%
%%                                          %%
%% The Main Body begins here                %%
%%                                          %%
%% Please refer to the instructions for     %%
%% authors on:                              %%
%% http://www.biomedcentral.com/info/authors%%
%% and include the section headings         %%
%% accordingly for your article type.       %%
%%                                          %%
%% See the Results and Discussion section   %%
%% for details on how to create sub-sections%%
%%                                          %%
%% use \cite{...} to cite references        %%
%%  \cite{koon} and                         %%
%%  \cite{oreg,khar,zvai,xjon,schn,pond}    %%
%%  \nocite{smith,marg,hunn,advi,koha,mouse}%%
%%                                          %%
%%%%%%%%%%%%%%%%%%%%%%%%%%%%%%%%%%%%%%%%%%%%%%

%%%%%%%%%%%%%%%%%%%%%%%%% start of article main body
% <put your article body there>


%% Background %%
\section*{Background}

Earlier work has shown how to optimally combine a set of
predictors---classifier or probability machines---into a so-called
{\it regression collective}~\cite{biau:2013}.  Consider, for example, a
collection of statistical learning machines chosen by the researcher,
from subject matter knowledge or with statistical aspirations. It
could contain versions of SVMs~\cite{vapnik:1998} with different
kernels, variants of the lasso~\cite{tibshirani:1996}, a group of
neural nets~\cite{ripley:1996}, a collection of $k$-nearest
neighbors~\cite{cover:hart:1967} with varying $k$, along with several
random forests~\cite{breiman:2001}. The thought here is that each or
several of these machines might be optimal for the data at hand, but
tuning each and adjudicating the multiple outcomes and performances
introduces a second layer of statistical and data analytic overhead.
Instead, a prefered way to proceed is to use a regression
collective which optimally combines machines, thus avoiding the
difficulty of individual machine tuning.

This new method~\cite{biau:2013}, whose R code is abbreviated COBRA
(for COmBined Regression Alternative), is just this kind of combining
method.  It has the property, that in the limit of large data, it is
at least as good as the best predictor in the collection, and
generates its prediction without having to declare which of the
individual predictors might be optimal for the data at hand. For
example, on a given data set one method might be Bayes optimal for
classification given enough data, and on another data set another
machine might be optimal. In all cases the regression collective,
given enough data, is Bayes optimal if any one machine is so, and
where this unnamed optimal machine in the portfolio of the collective
can vary over data sets.  The method achieves this optimality for any
data, with a possibly large number of features with mixed category or
continuous features and arbitrary correlation structure. As a
practical matter the regression collective requires no tuning of the
individual machines on the part of the researcher: the method is
entirely nonparametric and model-free. As one detail, there is no
requirement for specific and correct interaction terms, however
defined, as input for the method. Instead, machines that include such
interaction terms can be added to the portfolio of the collective.


The COBRA method is not a committee or ensemble method, nor is it a
voting method.  It is closer to a $k$-nearest neighbor scheme. However
the distance function, or metric, for measuring closeness in the
collective is not Euclidean distance or any weighting thereof, but a
method that uses the multiple predictions of the several component
machines to access closeness of a test data point to the training
data.  For each training data point, one checks if its predicted value
under a given machine is close to the predicted value of the test data
point under that same machine, and if closeness holds for a majority
of the machines, then that training point is deemed close to the
target test data point, otherwise it is deemed distant.  The final
prediction for the test data point is a sum of the training data
outcomes using only those data points that are close.  In particular
this means predictions from the several machines are not averaged to
make the prediction on the test case, but rather the predicted value
is a weighted average of the original outcomes.  That is, COBRA is a
type of locally weighted averaged estimator.

While the example above of a regression collective over a set of
learning machines was the motivation for COBRA~\cite{biau:2013}, it
has also lead to increased scrutiny of analytic approaches using
collections of features, biologically grounded networks or pathways
each as new inputs to other machines. Here the separate networks are
used as inputs to the various machines within the
collective~\cite{pan:2014}. Then, using the machines built from these
networks as {\it synthetic features}, they can be sent to a suitable
learning machine for which it is then possible to compare and evaluate
the predictive capacity and interactions between the networks: Are
some networks better than others in the portfolio? In what subset of
the data might that be true?

The approach described here links these two methods, that of a
regression collective and the introduction of synthetic features.  We
describe a {\it synthetic machine approach}, in particular an approach
we call {\it synthetic random forests}.  Using a collection of
differently tuned random forests, each generating a synthetic feature,
a synthetic random forest is defined as a secondary random forest, a
kind of hyperforest, calculated using the new input synthetic
features, along with all the original features.  The motivation for
using random forests as a combiner is motivated by the COBRA approach.
Like COBRA, a random forest can be described as a locally weighted
averaged estimator, however it differs from COBRA in that the weights
used to average training outcomes are arbitrary convex weights,
whereas COBRA weights are either zero or one values. It is the greater
flexibility afforded by convex weights that is the rationale for
considering random forests as a combiner.  We study the properties of
this new synthetic forest method using large scale simulations
involving both real and synthetic data. We find the method has the
similar property to COBRA that it appears to be as universally as
good, across all our test data sets, as the optimal machine in the
portfolio of its collective.  But not only that, our empirical
findings also suggest that the synthetic random forest outperforms the
orginal COBRA regression collective scheme.

%% Methods %%
\section*{Methods}

Random Forests~\cite{breiman:2001} (hereafter abbreviated as RF), is
an ensemble learning method which calculates ensemble predicted value
by aggregating a collection of $\ntree\ge 1$ randomly grown trees.  In
multiclass problems, averaging the terminal node relative frequency of
class labels over a forest of random classification trees yields
ensemble predicted probabilities for each class label, while in
regression problems, averaging the terminal node mean value yields
ensemble predicted values for the $Y$-response.  Equivalently, one can
show that the resulting ensemble predicted value of a RF can be
written as weighted convex combination of outcomes.  Thus, RF is a
locally weighted averaging estimator.  A unique feature of RF trees are
that they are random in the following sense: a) Each tree is grown
using an independent bootstrap sample (i.e.\ a sample drawn with
replacement from the original data set, of size $n$ equal to the
original sample size); (b) Random feature selection is employed in
which at each node of the tree during the tree growing process, a
random subset of $1\le \mtry\le p$ features are selected, where $p$
equals the total number of features, and the node is split using the
variable from the $\mtry$ candidate variables having the best split.
Splitting of a RF tree is repeated recursively, with the tree grown as
far as possible until it is no longer possible to identify groups that
differ on the outcome, or the sample size at that node is too small.
Terminal node sizes (the ends of the tree) satisfy the condition that
they contain a minimum of $\nodesize\ge 1$ unique cases.

Of the three tuning parameters used by RF, $(\ntree, \mtry,
\nodesize)$, optimal tuning of $\nodesize$ has the greatest potential
to improve prediction performance.  This is because $\nodesize$ acts
as a type of bandwidth parameter that controls the level of smoothing
of the RF predictor.  It has now become apparent to the machine
learning community that the optimal choice for $\nodesize$ depends
heavily on the underyling data.  In large $n$ sample settings for
example, it is generally believed that $\nodesize$ should be large to
ensure good performance.  Rationale for this comes from large sample
asymptotics which require $\nodesize$ to increase to $\infty$ in order
to ensure consistency.  Results of this nature have for example
been used to establish Bayes-risk consistency for RF
classification~\cite{Biau:Devroye:Lugosi:2008}.  On the other hand, in
high-dimensional problems involving a large number of features, the
opposite has been observed, with performance generally improving with
decreasing $\nodesize$~\cite{ishwaran:2011}.  In studying lower bounds
for the rate of convergence in RF regression, it has been shown that
rates of convergence improve when $\nodesize$ is small when the number
of features $p$ exceeds the sample size $n$~\cite{lin:jeon:2006}.

It is not hard to imagine settings where the underlying target
function $f$ of interest has curvature that varies over the feature
space.  Therefore, given that $\nodesize$ functions as a type of
bandwidth smoothing parameter, it stands to reason that an adaptive
$\nodesize$ value that becomes large or small depending upon the
flatness or wiggliness of $f$ will yield a RF that has the potential
to outperform a conventional forest constructed using a single fixed
$\nodesize$ value.  In order to allow RF to achieve this type of local
adaptivity, our idea is to create synthetic features, which themselves
are constructed from forests calculated using different $\nodesize$
values.  This then allows node splits of a synthetic RF tree to make
local and adaptive decisions about $\nodesize$ by selecting from
features constructed from different $\nodesize$ values.  It is this
key observation that forms the basis of the synthetic random forest
(SRF) method described below in Algorithm~\ref{A:SRF}.

\vskip6pt
\begin{algorithm}[pht]
\centering
\caption{\em\,\, Synthetic Random Forests (SRF)}\label{A:SRF}
\begin{algorithmic}[1]
\STATE Choose a set of candidate nodesize values $\nn=\{n_1,n_2,\ldots,n_D\}$.
\STATE Fit a RF with $\nodesize=n_j$ for $j=1,\ldots,D$.  Use the same
$\ntree$ and $\mtry$ value for each forest.  Denote the
resulting forests by $\RF_1,\ldots,\RF_D$.
\STATE Calculate the predicted value for each random forest $\RF_j$,
$j=1,\ldots,D$.  We call the predicted value the synthetic feature.
\STATE Fit a RF using for features both
 the newly created synthetic features and the
original $p$ features (using the same $\ntree$ and $\mtry$ value as
before).  We call this the synthetic RF.
\end{algorithmic}
\end{algorithm}

\noindent{\bf Remarks}\\ Implementing SRF conveniently involves doing
nothing more than fitting a RF to a slightly expanded set of features,
which includes in addition to the original $p$ features, a new
collection of synthetic features obtained by fitting RF under
different $\nodesize$ values.  While conceptually straightforward, there
are some important points to keep in mind when implementing SRF:

\vskip5pt
\begin{enumerate}
\item
The dimension of a synthetic feature can be one or greater.  In
regression, the synthetic feature is the predicted value of the
$Y$-response, which is one-dimensional, however in multiclass problems,
the synthetic feature is the predicted probability of the class
label.  If there are $J$ classes, this yields a $J$-dimensional
synthetic feature.  Note that since the predicted probabilities are
linearly dependent as they sum to 1, we discard by convention the last
coordinate and use only the first $J-1$ predicted probabilities.
\item
To avoid overfitting, when constructing the synthetic feature, we use
out-of-bag (OOB) predicted values.  In a bootstrap sample only 63.2\%
of the data is used on average (due to sampling with replacement),
leaving 36.7\% of the data untouched.  This latter data is termed OOB
because it is out of sample and can be used to calculate OOB ensemble
predicted values.  The OOB predicted value for each data point $\X$ does
not use the $Y$-response for $\X$ and therefore represents a cross-validated
out-of-sample estimate.
\item
The values of $\ntree$ and $\mtry$ are kept fixed throughout.
Selecting a reasonable value for $\ntree$ is not difficult and
performance is robust to its choice---as long as its value is kept
reasonably large, say 250 or more.  Optimizing over $\mtry$ can
improve RF but we have found that creating synthetic features by
varying both $\mtry$ and $\nodesize$ values can sometimes negatively
impact performance of SRF.  We find keeping $\mtry$ fixed at default
values and constructing synthetic features by varying $\nodesize$
works very well.
\item
Another reason for favoring optimization of $\nodesize$ rather than
$\mtry$ is its granularity.  For $\nodesize$, regardless of $n$ or
$p$, it suffices to consider a handful of small values, a few
intermediate values, and a few large values in the optimization (in
the case of large $n$, the rate at which $\nodesize$ converges to
$\infty$, required for consistency, can be far slower than $n$; thus
relatively small values of $\nodesize$ can be used even when $n$ is
relatively large).  In contrast, optimization over $\mtry$ depends
upon $p$, which creates not only an expensive optimization problem in
high-dimensions, but also the potential for overfitting due to the
addition of a large number of synthetic features.

\end{enumerate}

%% Results %%
\section*{Results}

We compared the performance of four methods, RF, \RFopt, SRF, and COBRA
over a collection of regression and multiclass benchmark datasets.
The four methods were defined as follows:

\begin{enumerate}
\item

SRF denotes Synthetic Random Forests described in Algorithm~1.  Values
for $\nodesize$ were set at
$\nn=\{1,2,3,4,5,6,7,8,9,10,20,30,50,100\}$.  The synthetic RF of line
4 was calculated using $\nodesize=5$.  While this value can be
included as a user parameter for SRF, we found that changing its value
did not alter our findings very much.  Thus we chose not to cloud our
findings and instead opted for a fixed value of $\nodesize=5$ throughout our
simulations.
\item
RF denotes a standard forest calculated using $\nodesize=5$.  The same
$\nodesize$ value was used as for the synthetic forest in SRF in order to
assess the efficacy of the synthetic features.  If the synthetic
features are not used in node splitting of a synthetic forest, the
resulting forest should closely approximate a regular forest, and thus
performance of SRF should closely approximate performance of RF.
\item
\RFopt\, denotes the forest calculated using the optimal nodesize from
$\nn$.  Specifically: the optimal nodesize was defined as the nodesize
value $n_j$ from the $\RF_j$ forest with the smallest OOB error in
SRF.  We include \RFopt\ to assess whether a globally nodesize-optimized
forest can compete with the locally nodesize-optimized synthetic
forest.
\item
COBRA implements the aggregation method described in~\cite{biau:2013}.
For regression machines required as input to COBRA we used the same
$\{\RF_j\}_1^D$ machines used by SRF.  Using the same synthetic
features as SRF allows us to assess the effectiveness of arbitrary
convex combination weighting used by RF compared with zero-one
weighting used by COBRA.  As a side note, we also tried implementing
COBRA using the default machines that comes with its code (lasso,
ridge regression, SVM and random forests) to assess whether a generic
COBRA implementation compared favorably to SRF.  However, the results
were so unfavorable that we excluded them from our findings.
\end{enumerate}

\vskip10pt
\noindent
All forests were calculated using $\ntree=500$ and $\mtry=[p/3]$ where
$[z]$ denotes the first integer greater than $z$.  Forest computations
were implemented using the R-package {\ttfamily
  randomForestSRC}~\cite{ishwaran:kogalur:2014} which has been
extended to include the function {\ttfamily rfsrcSyn} which implements
the Synthetic Random Forests described in Algorithm~1.  We note that
while forest calculations could have been implemented using other
random forest packages, such as {\ttfamily
  randomForest}~\cite{liaw2002classification}, we prefer to use
{\ttfamily randomForestSRC} as it has many useful features for
reducing computational times, such as parallel processing using the
OpenMP protocol (which we employed), and non-deterministic random
splitting via its \nsplit option (however while this option is
available in the {\ttfamily rfsrcSyn} function, it was not used here
to avoid clouding the issue of tuning parameters).  COBRA was
implemented using the R-package {\ttfamily COBRA}~\cite{guedj:2013}.
Calibration of the COBRA $\epsilon$-parameter which is recommended to
improve performance was implemented by selecting a grid consisting of
200 points.  Note that because the {\ttfamily COBRA} package has not
yet been extended to encompass multiclass problems, the COBRA method
was excluded from our multiclass experiment.

\vskip10pt\noindent{\bf Regression results}

\noindent A large collection of regression datasets was used to assess the
performance of each method (Table~1).  Datasets with a capital
identify real data while those in lower case are synthetic data.
Many of the synthetic data were obtained from the {\ttfamily mlbench}
R-package~\cite{mlbench:2012} and are labeled starting with ``mlb.''.
In total, 46 datasets were used with sample sizes varying from $n=31$
to $n=1114$; number of features varied from $p=2$ to $p=500$.
Sample sizes for synthetic data were set at $n=250$.

Performance was assessed using standardized mean-squared error (MSE)
defined as MSE divided by the variance of the $Y$-response and
multiplied by 100.  Standarized MSE facilitates comparison across
datasets: a value of 100 can be used as a benchmark value.  For real
data, MSE was calculated using 10-fold cross-validation.  For
synthetic data, MSE was evaluated by using an independent test-set of
size $n=5000$.  The entire process was repeated independently 100 times.
Table~1 reports the averaged standardized MSE from the 100 replicates.
Figure~1 displays the 95\% confidence regions of standardized MSE.

Table~1 and Figure~1 show clear superiority of SRF, especially over
synthetic data.  To formally assess performance differences we used
univariate and multivariate nonparametric statistical
tests~\cite{Demsar:2006}.  To compare two methods we used the Wilcoxon
signed rank test applied to the difference of their standardized MSE
values.  The exact p-value for the Wilcoxon signed rank test are
recorded along the upper diagonals of Table~2.  The lower diagonal
values record the corresponding test statistic where small values
indicate a difference.  To test for an overall difference among
procedures we used the Iman and Davenport modified Friedman
test~\cite{Demsar:2006}.  For each dataset, the performance of each
method was ranked from 1 through 4, and the average of these ranks
over all datasets for each procedure calculated.  The diagonal values
of the table record this average rank which was used for the Friedman
test.  This latter test yielded a near-zero p-value, thus providing
strong evidence of difference between methods.  Overall, SRF is ranked
first, followed by \RFopt, COBRA, and then RF.  Wilcoxon p-values
provide strong evidence supporting superiority of SRF to each of the
three other methods.


\vskip10pt\noindent{\bf Multiclass results}

\noindent To further assess the performance of SRF, a total of 38
multiclass benchmark datasets were used.  Sample sizes ranged from
$n=29$ to $n=6435$; features varied from $p=2$ to $p=8740$; and number
of classes $J$ varied from $J=2$ to $J=15$ (Table~3).  The same
nomenclature was adopted as in our regression experiment.  Real
datasets are indicated with capitals and synthetic data
from~{\ttfamily mlbench} are labeled starting with ``mlb.''.  Datasets
``aging'', ``brain'', ``colon'', ``leukemia'', ``lymphoma'' and
``srbct'' are well-known benchmark microarray datasets (note how $p\gg
n$ in each of these).

Note first that if a nonparametric regression scheme of any type,
learning machine or otherwise, is consistent for the expectation of
the outcome, then in a binary or multiclass group membership
prediction problem, it necessarily and automatically returns a
consistent estimate for the true conditional probability of group
membership. Hence, it makes sense to apply a standard measure of
probability estimation, so performance was assessed using the
classical Brier score (multiplied by 100). The Brier score directly
measures accuracy in estimating the true conditional probability, and
this is the task of any regression scheme given binary or multiclass
group membership.  As calibration, we note that a Brier score of 25
represents a procedure with performance no better than random
guessing.  As in the regression experiment, 10-fold validation was
used to estimate performance over real datasets and for synthetic data
an independent test-set of size $n=5000$ was used.  The entire process
was repeated independently 100 times.

Table~3 and Figure~2 show superiority of SRF to the three other
methods.  As in the regression experiment, performance differences are
especially noticeable over synthetic data.  Noticeable performance
differences are also observed over certain microarray datasets (srbct,
prostate, and leukemia).  Table~4 displays results of nonparametric
tests comparing procedures.  The results parallel those of Table~2:
SRF has best overall rank and there is strong evidence of its
superiority.  The modified Friedman test of equality of procedures
yielded a near zero p-value, further confirming evidence of SRF's
superior performance.

%% Conclusions %%
\section*{Conclusions}

Peering more closely at synthetic forests it is possible to discern a
reason for the generally good performance of any RF. That is, a single
RF is acting as a synthetic machine across all the features, where
each original feature is effectively a stand-alone synthetic feature.
The manner in which RF synthesizes its features also plays a vital
role in its success.  RF forms its predictor by taking a locally
weighted convex combination of the outcomes.  Importantly, this
differs from the COBRA method, which locally weights the outcomes
using zero-one weights.  The superior performance of synthetic forests
to COBRA found in our experiments, even when using the same synthetic
features as individual, separate constituents in the collective
portfolio, suggests that the use of convex, locally determined weights
may play a key role in its success, and where these weights are chosen
by the refined cells in the data space that are given by the terminal
nodes in each tree in each forest.  

Performance gains for synthetic forests were most noticeable among the
simulated data in our benchmark experiments. We believe the reason for
this is that these particular data structures have high signal and
sparse solutions.  The synthetic random forest, by varying the
synthetic inputs over a wide range of user-specified terminal node
sizes, acts as a local smoothing optimizer.  Our results suggest such
tuning is better able to handle high signal, sparse data.  Indeed,
especially noteworthy given this outcome, is that such data are known
to be especially challenging and are likely to constitute a
significant fraction of increasingly available big data sets.  Another
important practical implication of synthetic forests is that the
number of RF user tuning parameters are greatly minimized.  Most
importantly, the synthetic forest does this with evidently no loss in
prediction compared to a well-optimized single random forest.

Finally, and more comprehensively, the results here suggest that any
statistical learning machine, Super X say, that has user tuning
parameters, or indeed required parameter estimation, can be deployed
as a Synthetic Super X using RF, with less overhead and likely
no real loss in predictive capacity over the fully optimized Super X
on the given data.





%%%%%%%%%%%%%%%%%%%%%%%%%%%%%%%%%%%%%%%%%%%%%%
%%                                          %%
%% Backmatter begins here                   %%
%%                                          %%
%%%%%%%%%%%%%%%%%%%%%%%%%%%%%%%%%%%%%%%%%%%%%%

\begin{backmatter}

\section*{List of abbreviations}
COBRA: COmBined Regression Alternative.\\
RF: Random Forests.\\
\RFopt: Nodesize optimized random forests.\\
SRF: Synthetic Random Forests.

\section*{Competing interests}
The authors declare that they have no competing interests.

\section*{Author's contributions}

HI helped conceive the problem, developed the solution, provided
insight into synthetic machines, designed and implemented the
benchmark studies, and helped draft the manuscript.  JDM helped
conceive the problem, developed the solution, provided insight into
synthetic machines, and helped draft the manuscript.  All authors read
and approved the final manuscript.

\section*{Acknowledgements}
HI was funded by DMS grant 1148991 from the National Science
Foundation and grant R01CA163739 from the National Cancer Institute.
JDM was supported by the Intramural Research Program at the National
Institutes of Health.  The authors thank the referee of the paper for
their wonderfully helpful comments.


%%%%%%%%%%%%%%%%%%%%%%%%%%%%%%%%%%%%%%%%%%%%%%%%%%%%%%%%%%%%%
%%                  The Bibliography                       %%
%%                                                         %%
%%  Bmc_mathpys.bst  will be used to                       %%
%%  create a .BBL file for submission.                     %%
%%  After submission of the .TEX file,                     %%
%%  you will be prompted to submit your .BBL file.         %%
%%                                                         %%
%%                                                         %%
%%  Note that the displayed Bibliography will not          %%
%%  necessarily be rendered by Latex exactly as specified  %%
%%  in the online Instructions for Authors.                %%
%%                                                         %%
%%%%%%%%%%%%%%%%%%%%%%%%%%%%%%%%%%%%%%%%%%%%%%%%%%%%%%%%%%%%%

% if your bibliography is in bibtex format, use those commands:
\bibliographystyle{bmc-mathphys} % Style BST file
\bibliography{bmc_article}      % Bibliography file (usually '*.bib' )

% or include bibliography directly:
% \begin{thebibliography}
% \bibitem{b1}
% \end{thebibliography}

%%%%%%%%%%%%%%%%%%%%%%%%%%%%%%%%%%%
%%                               %%
%% Figures                       %%
%%                               %%
%% NB: this is for captions and  %%
%% Titles. All graphics must be  %%
%% submitted separately and NOT  %%
%% included in the Tex document  %%
%%                               %%
%%%%%%%%%%%%%%%%%%%%%%%%%%%%%%%%%%%

%%
%% Do not use \listoffigures as most will included as separate files

  \begin{figure}[ht]
  \caption{\csentence{Regression Benchmark Results.}  Cross-validated
    and test-set standardized mean-squared error (MSE) performance
    over 100 independent replications.   Boxplots display results
    from the 100 replications for COBRA
    (\gray{\blacksquare}), RF (\red{\blacksquare}), optimized random
    forests \RFopt\, (\blue{\blacksquare}), and synthetic random forests
    SRF (\black{\blacksquare}).  Standardized MSE obtained by
    dividing MSE by the variance of the $Y$-response and multiplying by
    100.}
  \begin{center}
    \resizebox{5.0in}{!}{\includegraphics{largeScaleBenchmark_regression.pdf}}
  \end{center}
  \end{figure}

  \begin{figure}[ht]
  \caption{\csentence{Multiclass Benchmark Results.}  Cross-validated
    and test-set Brier score performance ($\times 100$)
    over 100 independent replications.  Boxplots display results
    from the 100 replications for
    RF (\red{\blacksquare}), optimized random
    forests \RFopt\, (\blue{\blacksquare}), and synthetic random forests
    SRF (\black{\blacksquare}).}
  \begin{center}
    \resizebox{5.0in}{!}{\includegraphics{largeScaleBenchmark_multiclass.pdf}}
  \end{center}
  \end{figure}





%%%%%%%%%%%%%%%%%%%%%%%%%%%%%%%%%%%
%%                               %%
%% Tables                        %%
%%                               %%
%%%%%%%%%%%%%%%%%%%%%%%%%%%%%%%%%%%

%% Use of \listoftables is discouraged.
%%
\newpage

\begin{table}[pht]
\caption{Regression benchmark performance.
  Cross-validated
    and test-set standardized mean-squared error (MSE) performance
    over 100 independent replications.  Standardized MSE obtained by
    dividing MSE by the variance of the $Y$-response and multiplying by
    100.}
\centering
\begin{tabular}{rrrrrrr}
  \cline{2-7}
&$n$ & $p$ & COBRA & RF & \RFopt & SRF \\ 
  \cline{2-7}
Air & 111 & 5 & 27.24 & 28.68 & 27.53 & 28.14 \\ 
  Air2 & 111 & 5 & 28.40 & 30.72 & 28.85 & 28.36 \\ 
  Automobile & 193 & 29 & 9.83 & 8.94 & 6.79 & 7.52 \\ 
  Bodyfat & 252 & 13 & 31.36 & 32.02 & 31.67 & 32.19 \\ 
  BostonHousing & 506 & 13 & 18.88 & 14.64 & 12.39 & 12.80 \\ 
  BostonHousing2 & 506 & 16 & 17.44 & 13.57 & 11.32 & 11.61 \\ 
  CMB & 899 & 4 & 96.33 & 100.90 & 90.32 & 89.86 \\ 
  Crime & 47 & 15 & 61.74 & 59.99 & 59.51 & 59.03 \\ 
  Diabetes & 442 & 10 & 57.58 & 53.22 & 53.14 & 55.20 \\ 
  DiabetesI & 442 & 64 & 57.05 & 54.42 & 54.61 & 55.92 \\ 
  Fitness & 31 & 6 & 83.34 & 66.48 & 59.61 & 57.76 \\ 
  Highway & 39 & 11 & 38.84 & 43.67 & 33.95 & 32.18 \\ 
  Iowa & 33 & 9 & 62.60 & 62.16 & 50.03 & 50.22 \\ 
  Ozone & 203 & 12 & 26.90 & 26.19 & 26.20 & 26.42 \\ 
  OzoneI & 203 & 134 & 27.42 & 26.14 & 26.32 & 26.08 \\ 
  Pollute & 60 & 15 & 49.64 & 51.36 & 49.52 & 46.74 \\ 
  Prostate & 97 & 8 & 87.32 & 46.02 & 46.95 & 50.12 \\ 
  Servo & 167 & 19 & 15.22 & 21.47 & 11.27 & 11.99 \\ 
  ServoFactor & 167 & 16 & 43.24 & 34.65 & 32.54 & 31.44 \\ 
  Tecator & 215 & 22 & 13.84 & 16.11 & 13.48 & 6.19 \\ 
  Tecator2 & 215 & 100 & 31.24 & 34.21 & 30.64 & 27.94 \\ 
  Windmill & 1114 & 12 & 31.64 & 31.39 & 31.31 & 32.15 \\ 
  expon & 250 & 2 & 47.76 & 46.04 & 46.48 & 47.60 \\ 
  expon.noise & 250 & 17 & 62.13 & 67.49 & 66.44 & 53.04 \\ 
  mlb.friedman1 & 250 & 10 & 21.46 & 26.11 & 24.15 & 19.04 \\ 
  mlb.friedman1.noise & 250 & 10 & 30.91 & 34.77 & 33.13 & 30.48 \\ 
  mlb.friedman1.bigp & 250 & 250 & 37.67 & 44.14 & 43.81 & 31.99 \\ 
  mlb.friedman2 & 250 & 4 & 13.94 & 14.75 & 14.24 & 14.04 \\ 
  mlb.friedman2.noise & 250 & 4 & 37.19 & 36.77 & 36.80 & 38.58 \\ 
  mlb.friedman2.bigp & 250 & 254 & 22.92 & 29.01 & 28.10 & 17.73 \\ 
  mlb.friedman3 & 250 & 4 & 19.21 & 22.01 & 19.87 & 15.59 \\ 
  mlb.friedman3.noise & 250 & 4 & 37.47 & 39.38 & 38.53 & 36.97 \\ 
  mlb.friedman3.bigp & 250 & 254 & 37.19 & 46.72 & 45.47 & 26.78 \\ 
  mlb.peak & 250 & 20 & 14.75 & 17.24 & 16.28 & 6.21 \\ 
  mlb.peak.bigp & 250 & 20 & 14.75 & 17.24 & 16.28 & 6.21 \\ 
  mlb.noise & 250 & 500 & 101.69 & 100.75 & 100.47 & 100.29 \\ 
  sine & 250 & 2 & 35.92 & 37.79 & 35.95 & 34.72 \\ 
  sine.noise & 250 & 5 & 56.64 & 66.07 & 61.14 & 54.71 \\ 
  syn.ex1 & 250 & 50 & 20.69 & 30.87 & 28.57 & 8.54 \\ 
  syn.ex2 & 250 & 20 & 88.60 & 89.66 & 89.59 & 92.68 \\ 
  syn.ex3 & 250 & 50 & 43.88 & 47.88 & 47.50 & 43.04 \\ 
  syn.ex4 & 250 & 50 & 34.75 & 37.78 & 36.90 & 30.40 \\ 
  syn.ex5 & 250 & 20 & 62.50 & 65.07 & 64.82 & 62.80 \\ 
  syn.ex6 & 250 & 30 &  & 102.30 & 100.58 & 103.16 \\ 
  syn.ex7 & 250 & 300 & 55.13 & 61.68 & 61.38 & 52.41 \\ 
  syn.ex8 & 250 & 50 & 117.93 & 58.11 & 57.76 & 52.01 \\ 
  \cline{2-7}
\end{tabular}
\end{table}


\vskip20pt
\begin{table}[htb!]
\caption{Regression benchmark performance.
  Upper diagonal values are Wilcoxon signed rank p-values comparing
  two procedures; lower diagonal values are the corresponding test
  statistic.  Diagonal values record the overall rank of a procedure.}
\centering
\begin{tabular}{rrrrr}
  \hline
 & COBRA & RF & \RFopt & SRF \\ 
  \hline
COBRA & \bf{2.7} & 0.0968 & 0.3703 & 0.0000 \\ 
  RF & 388& \bf{3.3} & 0.0000 & 0.0000 \\ 
  \RFopt & 623 & 1029 & \bf{2.3} & 0.0018 \\ 
  SRF & 1005 & 966 & 827 & \bf{1.7} \\ 
   \hline
\end{tabular}
\end{table}



\begin{table}[hbt]
\caption{Multiclass benchmark performance.
  Cross-validated and test-set Brier score performance ($\times 100$)
    over 100 independent replications.} 
\centering
\begin{tabular}{rrrrrrr}
  \cline{2-7}
& $n$ & $p$ & $J$ & RF & \RFopt & SRF \\ 
  \cline{2-7}
BreastCancer & 683 & 10 & 2 & 2.59 & 2.50 & 2.28 \\ 
  DNA & 3186 & 180 & 3 & 3.03 & 2.79 & 2.34 \\ 
  Esophagus & 3127 & 28 & 2 & 18.33 & 17.81 & 18.27 \\ 
  Glass & 214 & 9 & 6 & 6.16 & 6.20 & 5.78 \\ 
  HouseVotes84 & 232 & 16 & 2 & 5.85 & 4.82 & 4.41 \\ 
  Hypothyroid & 2000 & 24 & 2 & 1.20 & 1.18 & 1.14 \\ 
  Ionosphere & 351 & 34 & 2 & 5.76 & 5.28 & 5.14 \\ 
  PimaIndiansDiabetes & 768 & 8 & 2 & 15.69 & 15.66 & 16.21 \\ 
  Prostate & 158 & 20 & 2 & 15.81 & 15.83 & 16.02 \\ 
  Satellite & 6435 & 36 & 6 & 2.30 & 1.98 & 1.92 \\ 
  SickEuthyroid & 2000 & 24 & 2 & 2.51 & 2.35 & 2.30 \\ 
  Sonar & 208 & 60 & 2 & 12.91 & 12.46 & 9.73 \\ 
  SouthAfricanHeart & 462 & 9 & 2 & 19.69 & 19.34 & 19.86 \\ 
  Soybean & 562 & 35 & 15 & 0.82 & 0.71 & 0.77 \\ 
  Spam & 4601 & 57 & 2 & 4.39 & 4.18 & 3.74 \\ 
  Vehicle & 846 & 18 & 4 & 7.51 & 8.81 & 6.82 \\ 
  Vowel & 990 & 10 & 11 & 2.66 & 1.81 & 1.09 \\ 
  WisconsinBreast & 699 & 10 & 2 & 3.13 & 3.05 & 3.07 \\ 
  Zoo & 101 & 16 & 7 & 1.53 & 0.51 & 1.30 \\ 
  aging & 29 & 8740 & 3 & 16.64 & 16.96 & 16.54 \\ 
  brain & 42 & 5597 & 5 & 8.32 & 7.07 & 7.99 \\ 
  colon & 62 & 2000 & 2 & 12.88 & 12.78 & 12.78 \\ 
  leukemia & 72 & 3571 & 2 & 4.06 & 3.95 & 2.45 \\ 
  lymphoma & 62 & 4026 & 3 & 2.71 & 2.62 & 2.31 \\ 
  prostate & 102 & 6033 & 2 & 8.36 & 8.25 & 5.85 \\ 
  srbct & 63 & 2308 & 4 & 3.62 & 3.70 & 2.45 \\ 
  mlb.cassini & 250 & 2 & 3 & 0.92 & 0.55 & 0.62 \\ 
  mlb.circle & 250 & 2 & 2 & 5.27 & 4.59 & 4.26 \\ 
  mlb.cuboids & 250 & 3 & 4 & 0.66 & 0.53 & 0.57 \\ 
  mlb.dnormals & 250 & 2 & 2 & 6.24 & 6.30 & 6.31 \\ 
  mlb.ringnorm & 250 & 20 & 2 & 10.71 & 10.17 & 4.83 \\ 
  mlb.shapes & 250 & 2 & 4 & 0.87 & 0.70 & 0.52 \\ 
  mlb.smiley & 250 & 2 & 4 & 0.51 & 0.26 & 0.58 \\ 
  mlb.spirals & 250 & 2 & 2 & 1.66 & 0.72 & 0.18 \\ 
  mlb.threenorm & 250 & 20 & 2 & 15.62 & 15.34 & 12.98 \\ 
  mlb.twonorm & 250 & 20 & 2 & 8.50 & 7.87 & 4.31 \\ 
  mlb.waveform & 250 & 21 & 3 & 9.34 & 9.47 & 7.83 \\ 
  mlb.xor & 250 & 2 & 2 & 3.61 & 2.50 & 1.30 \\ 
  \cline{2-7}
\end{tabular}
\end{table}

\begin{table}[hbt!]
\caption{Multiclass benchmark performance.
  Upper diagonal values are Wilcoxon signed rank p-values comparing
  two procedures; lower diagonal values are the corresponding test
  statistic.  Diagonal values record the overall rank of a procedure.}
\centering
\begin{tabular}{rrrr}
  \hline
 & RF & \RFopt & SRF \\ 
  \hline
  RF & \bf{2.68} & 0.0000 & 0.0000 \\ 
  \RFopt & 648 & \bf{1.86} & 0.0045 \\ 
  SRF & 688 & 563 & \bf{1.45} \\ 
   \hline
\end{tabular}
\end{table}

%%%%%%%%%%%%%%%%%%%%%%%%%%%%%%%%%%%
%%                               %%
%% Additional Files              %%
%%                               %%
%%%%%%%%%%%%%%%%%%%%%%%%%%%%%%%%%%%

%\section*{Additional Files}
%  \subsection*{Additional file 1 --- Sample additional file title}
%    Additional file descriptions text (including details of how to
%    view the file, if it is in a non-standard format or the file extension).  This might
%    refer to a multi-page table or a figure.

%  \subsection*{Additional file 2 --- Sample additional file title}
%    Additional file descriptions text.


\end{backmatter}
\end{document}


